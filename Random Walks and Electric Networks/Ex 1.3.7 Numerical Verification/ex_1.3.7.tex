\documentclass{article}\usepackage[]{graphicx}\usepackage[]{color}
% maxwidth is the original width if it is less than linewidth
% otherwise use linewidth (to make sure the graphics do not exceed the margin)
\makeatletter
\def\maxwidth{ %
  \ifdim\Gin@nat@width>\linewidth
    \linewidth
  \else
    \Gin@nat@width
  \fi
}
\makeatother

\definecolor{fgcolor}{rgb}{0.345, 0.345, 0.345}
\newcommand{\hlnum}[1]{\textcolor[rgb]{0.686,0.059,0.569}{#1}}%
\newcommand{\hlstr}[1]{\textcolor[rgb]{0.192,0.494,0.8}{#1}}%
\newcommand{\hlcom}[1]{\textcolor[rgb]{0.678,0.584,0.686}{\textit{#1}}}%
\newcommand{\hlopt}[1]{\textcolor[rgb]{0,0,0}{#1}}%
\newcommand{\hlstd}[1]{\textcolor[rgb]{0.345,0.345,0.345}{#1}}%
\newcommand{\hlkwa}[1]{\textcolor[rgb]{0.161,0.373,0.58}{\textbf{#1}}}%
\newcommand{\hlkwb}[1]{\textcolor[rgb]{0.69,0.353,0.396}{#1}}%
\newcommand{\hlkwc}[1]{\textcolor[rgb]{0.333,0.667,0.333}{#1}}%
\newcommand{\hlkwd}[1]{\textcolor[rgb]{0.737,0.353,0.396}{\textbf{#1}}}%
\let\hlipl\hlkwb

\usepackage{framed}
\makeatletter
\newenvironment{kframe}{%
 \def\at@end@of@kframe{}%
 \ifinner\ifhmode%
  \def\at@end@of@kframe{\end{minipage}}%
  \begin{minipage}{\columnwidth}%
 \fi\fi%
 \def\FrameCommand##1{\hskip\@totalleftmargin \hskip-\fboxsep
 \colorbox{shadecolor}{##1}\hskip-\fboxsep
     % There is no \\@totalrightmargin, so:
     \hskip-\linewidth \hskip-\@totalleftmargin \hskip\columnwidth}%
 \MakeFramed {\advance\hsize-\width
   \@totalleftmargin\z@ \linewidth\hsize
   \@setminipage}}%
 {\par\unskip\endMakeFramed%
 \at@end@of@kframe}
\makeatother

\definecolor{shadecolor}{rgb}{.97, .97, .97}
\definecolor{messagecolor}{rgb}{0, 0, 0}
\definecolor{warningcolor}{rgb}{1, 0, 1}
\definecolor{errorcolor}{rgb}{1, 0, 0}
\newenvironment{knitrout}{}{} % an empty environment to be redefined in TeX

\usepackage{alltt}
\IfFileExists{upquote.sty}{\usepackage{upquote}}{}
\begin{document}

\title{Random Walks and Electric Networks: Exercise 1.3.7 Numerical Verification}

\author{Chris Hayduk}
\date{March 4, 2021}

\maketitle



Define the transition probability matrix $p$ as

\begin{knitrout}
\definecolor{shadecolor}{rgb}{0.969, 0.969, 0.969}\color{fgcolor}\begin{kframe}
\begin{verbatim}
##           a         b         c         d         e         f         g
## a 1.0000000 0.0000000 0.0000000 0.0000000 0.0000000 0.0000000 0.0000000
## b 0.0000000 1.0000000 0.0000000 0.0000000 0.0000000 0.0000000 0.0000000
## c 0.3333333 0.3333333 0.0000000 0.0000000 0.3333333 0.0000000 0.0000000
## d 0.3333333 0.3333333 0.0000000 0.0000000 0.0000000 0.3333333 0.0000000
## e 0.0000000 0.0000000 0.3333333 0.0000000 0.0000000 0.0000000 0.3333333
## f 0.0000000 0.0000000 0.0000000 0.3333333 0.0000000 0.0000000 0.3333333
## g 0.3333333 0.0000000 0.0000000 0.0000000 0.3333333 0.3333333 0.0000000
## h 0.0000000 0.3333333 0.0000000 0.0000000 0.3333333 0.3333333 0.0000000
##           h
## a 0.0000000
## b 0.0000000
## c 0.0000000
## d 0.0000000
## e 0.3333333
## f 0.3333333
## g 0.0000000
## h 0.0000000
\end{verbatim}
\end{kframe}
\end{knitrout}

Then the probabilities of reaching a or b first from each state are given by the following matrix,
\begin{knitrout}
\definecolor{shadecolor}{rgb}{0.969, 0.969, 0.969}\color{fgcolor}\begin{kframe}
\begin{verbatim}
##           a         b c d e f g h
## a 1.0000000 0.0000000 0 0 0 0 0 0
## b 0.0000000 1.0000000 0 0 0 0 0 0
## c 0.5000000 0.5000000 0 0 0 0 0 0
## d 0.5000000 0.5000000 0 0 0 0 0 0
## e 0.5000000 0.5000000 0 0 0 0 0 0
## f 0.5000000 0.5000000 0 0 0 0 0 0
## g 0.6666667 0.3333333 0 0 0 0 0 0
## h 0.3333333 0.6666667 0 0 0 0 0 0
\end{verbatim}
\end{kframe}
\end{knitrout}

where we have raised $p$ to the power $10,000$. Since $a$ has a $1/3$ chance of transitioning to $c, d$ and $g$, we can write the probability that the chain will hit $b$ before returning to $a$ as $1/3 \cdot 0.5 + 1/3 \cdot 0.5 + 1/3 \cdot 0.33 \approx 0.44$
\end{document}
